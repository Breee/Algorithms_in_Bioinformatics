\section{Multiple Sequence-Alignment}
\label{sec:multi_alignment}

%\subsection{Needleman-Wunsch-Algorithm for n = 3 sequences}
%\label{sec:needleman_multi}
%
%\subsection{Sum-of-Pairs-Algorithm}
%\label{sec:sum_of_pairs}

\subsection{Feng-Doolittle-Algorithm}
\label{sec:feng_doolittle}
The Feng-Doolittle-Algorithm is used for progressive alignment and the workflow is as follows:
\begin{enumerate}
  \item INPUT: a set of sequences $S$.
  \item Perfom all pairwise sequence alignments using Needleman Wunsch or another alignment algorithm.
  \item Convert the scores of all alignments to evolutionary distances.
  \item Contruct a guide Tree.
  \item Compute the MSA by traversing the guide tree node root to the bottom
        and obtain a MSA. There are three cases which are handled:

        Case 1: Apply operation 1 on the sequences returned by the leafs and return the alignment.

        Case 2: Apply operation 2 on the sequence and the alignment and return the alignment.

        Case 3: Both Children are inner nodes, Apply operation 3 on the alignments and return the alignment.

        Operation 1 computes best pairwise alignment of two nodes and changes occurences of gap symbol to X.

        Operation 2 aligns a sequence S to an alignment A where we
        1. Compute pairwise alignment score between S and all sequences S' in A,
        2. Align S according to the sequence in the best pairwise alignment,
        3. Change occurences of gap symbol to X.

        Operation 3 aligns an alignment A1 to an alignment A2.
        For each pair of sequences S1 in A1 and S2 in A2 compute pairwise alignment score
        Align A1 and A2 according to the pairwise alignment with minimal distance.
        Change occurences of gap symbol to X

\end{enumerate}
