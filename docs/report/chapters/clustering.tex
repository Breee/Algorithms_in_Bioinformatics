%!TEX root = ../proposal.tex

\section{Phylogeny/Clustering}
\label{sec:clustering}

\subsection{UPGMA/WPGMA}
\label{sec:UPGMA}
We now explain UPGMA and WPGMA, based on \cite{durbin_eddy_krogh_mitchison_1998, DBLP:journals/almob/ApostolicoCDP13}.
Consider some set of sequences, which is shown in~\cref{fig:seq}.
\begin{figure}[H]
  \centering
    \begin{tabular}{ll}
      \textbf{A:} & TCAACTAC \\
      \textbf{B:} & ACTGCAAA \\
      \textbf{C:} & GGCTGTAA \\
      \textbf{D:} & AGTTGCAA \\
      \textbf{E:} & TTTGAACT
    \end{tabular}
  \caption{Example sequences}
  \label{fig:seq}
\end{figure}
The aim is \textcolor{ALUblue}{grouping} the most \textcolor{ALUblue}{similar sequences}, regardless of their evolutionary rate or phylogenetic affinities.
\textcolor{ALUblue}{U}nweighted / \textcolor{ALUblue}{W}eighted \textcolor{ALUblue}{P}air \textcolor{ALUblue}{G}roup \textcolor{ALUblue}{M}ethod with \textcolor{ALUblue}{A}rithmetic Mean (UPGMA/WPGMA)
is a  hierarchical clustering method used for the \textcolor{ALUblue}{creation of guide trees}.
The algorithm works as follows:
\begin{enumerate}
  \item Compute the \textcolor{ALUblue}{distance between each pair of sequences}.
  \item Treat \textcolor{ALUblue}{each sequence} as a \textcolor{ALUblue}{cluster $C$} by itself.
  \item \textcolor{ALUblue}{Merge the two closest clusters}. The distance between two clusters is the \textcolor{ALUred}{average distance between all their sequences}:
  $$d(C_i,C_j) = \frac{1}{|C_i| |C_j|} \sum_{r \in C_i, s\in C_j} d(r,s)$$
  where $C_i, C_j$ are clusters and $i \not= j$.
  \item Repeat 2. and 3. until only one cluster remains.
\end{enumerate}

In \textcolor{ALUblue}{WPGMA} the \textcolor{ALUblue}{distance between clusters} is calculated as a \textcolor{ALUblue}{simple average}:
$$d(C_i \cup C_j, C_k) = \frac{d(C_i,C_k) + d(C_j,C_k)}{2}$$
WPGMA is computationally easier than  \textcolor{ALUblue}{UPGMA}.
\textcolor{ALUred}{Unequal numbers of taxa} in the clusters cause problems, this means the distances in the original matrix \textcolor{ALUred}{do not contribute equally} to the intermediate calculations.
The branches do not preserve the original distances, the final result is therefore said to be \textcolor{ALUblue}{weighted}.

Clustering works only if the data are \textcolor{ALUblue}{ultrametric}.
Ultrametric distances are defined by the satisfaction of the \textcolor{ALUblue}{three-point condition}:
\begin{itemize}
  \item For any three taxa it holds:
  $$d(A, C) \leq max(d(A,B), d(B,C))$$
\end{itemize}

 So we assume that all taxa evolve with the same constant rate
The runtime is  $O(n^3)$ for the trivial approach, $O(n^2 log(n))$, when using a heap for each cluster
and $O(k 3^k n^2) / O(n^2)$ implementations for special cases. (by Fionn Murtagh,  by Day and Edelsbrunner)

\begin{example}[UPGMA example]
 Let $\textcolor{ALUblue}{A},\textcolor{ALUblue}{B},\textcolor{ALUblue}{C},\textcolor{ALUblue}{D},\textcolor{ALUblue}{E}$ be sequences.
 Say we have calculated a distance matrix $D$ using a pairwise alignment algorithm, which were explained in \cref{sec:pairwise_alignment}.
 $D$ is shown in~\cref{fig:ex1_1}
 \begin{figure}[H]
   \centering
   \begin{tabular}{c|c|c|c|c|c|}
   \cline{2-6}
                           & A & B & C & D & E \\ \hline
   \multicolumn{1}{|c|}{A} & 0 & 8 & 4 & 6 & 8 \\ \hline
   \multicolumn{1}{|c|}{B} & 8 & 0 & 8 & 8 & 4 \\ \hline
   \multicolumn{1}{|c|}{C} & 4 & 8 & 0 & 6 & 8 \\ \hline
   \multicolumn{1}{|c|}{D} & 6 & 8 & 6 & 0 & 8 \\ \hline
   \multicolumn{1}{|c|}{E} & 8 & 4 & 8 & 8 & 0 \\ \hline
   \end{tabular}
   \caption{Distance matrix $D$ of A,B,C,D,E }
   \label{fig:ex1_1}
 \end{figure}
We now want to create the guide tree for sequences A,B,C,D,E.
The sequences are from now on treated as clusters. $\{A\},\{B\}, \{C\}, \{D\}, \{E\}$
\begin{enumerate}
  \item \cref{fig:ex1_2} illustrates the initialization of the algorithm.
  We start with the whole distance matrix and create a guide tree, which consists just of leave nodes that represent the clusters.
\begin{figure}[H]
	\centering
	\begin{subfigure}{.4\textwidth}
    \centering
    \begin{tabular}{c|c|c|c|c|c|}
      \cline{2-6}
                              & A & B & C & D & E \\ \hline
      \multicolumn{1}{|c|}{A} & 0 & 8 & 4 & 6 & 8 \\ \hline
      \multicolumn{1}{|c|}{B} & 8 & 0 & 8 & 8 & 4 \\ \hline
      \multicolumn{1}{|c|}{C} & 4 & 8 & 0 & 6 & 8 \\ \hline
      \multicolumn{1}{|c|}{D} & 6 & 8 & 6 & 0 & 8 \\ \hline
      \multicolumn{1}{|c|}{E} & 8 & 4 & 8 & 8 & 0 \\ \hline
      \end{tabular}
	\end{subfigure}
%%%%%%%%%%%%%%
	\begin{subfigure}{.4\textwidth}
    \centering
    \begin{tikzpicture}[->,>=stealth',shorten >=1pt,auto,node distance=0.8cm,
        semithick,scale=1, every node/.style={scale=1}]
        \tikzstyle{every state}=[fill=white,draw=black,text=black,minimum size=5mm,inner sep=2pt]
        \node[state,align=center](A){A};
        \node[state,align=center](B)[right of=A]{B};
        \node[state,align=center](C)[right of=B]{C};
        \node[state,align=center](D)[right of=C]{D};
        \node[state,align=center](E)[right of=D]{E};
     \end{tikzpicture}
	\end{subfigure}
  \captionsetup{margin=2cm}
	\caption{Initialization. On the left side, we see the distance matrix $D$.
  On the right side we see the guide tree.}
  \label{fig:ex1_2}
\end{figure}

\item \cref{fig:ex1_3} illustrates this step of the algorithm.
 \textcolor{ALUblue}{\{A\}} and \textcolor{ALUblue}{\{C\}} are the closest clusters,
so the algorithm \textcolor{ALUgreen}{merges \{A\} and \{C\}  into \{A,C\}}.
In the guide tree a new node is added, and nodes $A$ and $C$ connected to it.
Each branch gets a total distance of $\frac{d(\{A\},\{C\})}{2} = \frac{4}{2}$.
\begin{figure}[H]
	\centering
	\begin{subfigure}{.4\textwidth}
    \centering
    \begin{tabular}{c|c|c|c|c|c|}
    \cline{2-6}
                                                                           & A & B & \cellcolor[HTML]{4B93C7}{\color[HTML]{EFEFEF} C} & D & E \\ \hline
    \multicolumn{1}{|c|}{\cellcolor[HTML]{4B93C7}{\color[HTML]{EFEFEF} A}} & 0 & 8 & \cellcolor[HTML]{F8A102}{\color[HTML]{000000} 4} & 6 & 8 \\ \hline
    \multicolumn{1}{|c|}{B}                                                & 8 & 0 & 8                                                & 8 & 4 \\ \hline
    \multicolumn{1}{|c|}{C}                                                & 4 & 8 & 0                                                & 6 & 8 \\ \hline
    \multicolumn{1}{|c|}{D}                                                & 6 & 8 & 6                                                & 0 & 8 \\ \hline
    \multicolumn{1}{|c|}{E}                                                & 8 & 4 & 8                                                & 8 & 0 \\ \hline
    \end{tabular}
	\end{subfigure}
%%%%%%%%%%%%%%
	\begin{subfigure}{.4\textwidth}
    \centering
    \begin{tikzpicture}[->,>=stealth',shorten >=1pt,auto,node distance=1.1cm,
        semithick,scale=1, every node/.style={scale=1}]
        \tikzstyle{every state}=[fill=white,draw=black,text=black,minimum size=3mm,inner sep=2pt]
        \node[state,align=center,fill=YellowOrange](A){A};
        \node[state,align=center, fill=YellowOrange](C)[right of=A]{C};
        \path (A) -- (C) coordinate[midway] (aux);
        \node[state,align=center, fill=YellowOrange, yshift=-0.5cm](AC)[below of=aux]{};
        \node[state,align=center](B)[right of=C]{B};
        \node[state,align=center](D)[right of=B]{D};
        \node[state,align=center](E)[right of=D]{E};
        \path (A) edge[draw=YellowOrange] node[anchor=center,align=center, draw=YellowOrange, fill=YellowOrange,rounded rectangle] {\scriptsize2} (AC);
        \path (C) edge[draw=YellowOrange] node[anchor=center,align=center, draw=YellowOrange, fill=YellowOrange,rounded rectangle] {\scriptsize2} (AC);
     \end{tikzpicture}
	\end{subfigure}
  \captionsetup{margin=2cm}
	\caption{Merge of \textcolor{ALUblue}{\{A\}} and \textcolor{ALUblue}{\{C\}}. On the left side, we see the distance matrix $D$.
  On the right side we see the guide tree,
  where the orange colored nodes branches represent the changes made to the guide tree.}
\label{fig:ex1_3}
\end{figure}
\item \cref{fig:ex1_4} illustrates the distance matrices, before and after the merge of $\{A\}$ and $\{C \}$.
The distance between $\{A,C\}$ and the other clusters is the \textcolor{ALUred}{average distance between all their sequences}.
Example: the new distance between $\{A,C\}$ and $\{B\}$ is:
  \begin{align*}
    d(\{A,C\}, \{B\}) &= \frac{1}{|\{A,C\}||\{B\}|}  (d(\{A\},\{B\}) + d(\{C\},\{B\}))\\
                      &= \frac{1}{2} (8+8) = 8
  \end{align*}
  \tikzset{box/.style={inner xsep=0pt}}
  \begin{figure}[H]
  	\centering
  	\begin{subfigure}{.4\textwidth}
      \centering
      \begin{tabular}{c|c|c|c|c|c|}
        \cline{2-6}
        & A & B & \cellcolor[HTML]{4B93C7}{\color[HTML]{EFEFEF} C} & D & E \\ \hline
        \multicolumn{1}{|c|}{\cellcolor[HTML]{4B93C7}{\color[HTML]{EFEFEF} A}} & 0 & 8 & \cellcolor[HTML]{F8A102}{\color[HTML]{000000} 4} & 6 & 8 \\ \hline
        \multicolumn{1}{|c|}{B}                                                & 8 & 0 & 8                                                & 8 & 4 \\ \hline
        \multicolumn{1}{|c|}{C}                                                & 4 & 8 & 0                                                & 6 & 8\tikzmark{a}\\ \hline
        \multicolumn{1}{|c|}{D}                                                & 6 & 8 & 6                                                & 0 & 8 \\ \hline
        \multicolumn{1}{|c|}{E}                                                & 8 & 4 & 8                                                & 8 & 0 \\ \hline
      \end{tabular}
  	\end{subfigure}
  %%%%%%%%%%%%%%
  	\begin{subfigure}{.4\textwidth}
      \centering
      \begin{tabular}{c|c|c|c|c|}
      \cline{2-5}
                                                        & \cellcolor[HTML]{F8A102}A,C & B & D & E \\ \hline
      \multicolumn{1}{|c|}{\cellcolor[HTML]{F8A102}A,C} & 0                           & 8 & 6 & 8 \\  \hline
      \multicolumn{1}{|c|}{\tikzmark{b}B}                           & 8                           & 0 & 8 & 4 \\ \hline
      \multicolumn{1}{|c|}{D}                           & 6                           & 8 & 0 & 8 \\ \hline
      \multicolumn{1}{|c|}{E}                           & 8                           & 4 & 8 & 0 \\ \hline
      \end{tabular}
  	\end{subfigure}
    \captionsetup{margin=2cm}
  	\caption{Recalulation of distances, after the merge of \textcolor{ALUblue}{\{A\}} and \textcolor{ALUblue}{\{C\}}.
    On the left side, we see the old distance matrix.
    On the right side we see the new distance matrix, where \textcolor{ALUblue}{\{A\}} and \textcolor{ALUblue}{\{C\}} are merged.}
  \label{fig:ex1_4}
  \end{figure}
  \begin{tikzpicture}[overlay,remember picture,->]
    \draw[very thick]         ($({pic cs:a})+(4.0ex,1ex)$)
        to   ($({pic cs:b})+(-4.8ex,-0.4ex)$);
    \end{tikzpicture}


\item \cref{fig:ex1_5} illustrates this step of the algorithm.
 \textcolor{ALUblue}{\{B\}} and \textcolor{ALUblue}{\{E\}} are the closest clusters,
such that the algorithm \textcolor{ALUgreen}{merges \{B\} and \{E\}  into \{B,E\}}.
In the guide tree a new node is added, and nodes $B,E$ are connected to it.
It assigns $\frac{d(\{B\},\{E\})}{2} = \frac{4}{2} = 2$ to each branch.
\begin{figure}[H]
	\centering
	\begin{subfigure}{.4\textwidth}
    \centering
    \begin{tabular}{c|c|c|c|c|}
    \cline{2-5}
                                                      & \cellcolor[HTML]{FFFFFF}A,C & B & D & \cellcolor[HTML]{4B93C7}E \\ \hline
    \multicolumn{1}{|c|}{\cellcolor[HTML]{FFFFFF}A,C} & 0                           & 8 & 6 & 8                         \\ \hline
    \multicolumn{1}{|c|}{\cellcolor[HTML]{4B93C7}B}   & 8                           & 0 & 8 & \cellcolor[HTML]{F8A102}4 \\ \hline
    \multicolumn{1}{|c|}{D}                           & 6                           & 8 & 0 & 8                         \\ \hline
    \multicolumn{1}{|c|}{E}                           & 8                           & 4 & 8 & 0                         \\ \hline
    \end{tabular}
	\end{subfigure}
%%%%%%%%%%%%%%
	\begin{subfigure}{.4\textwidth}
    \centering
    \begin{tikzpicture}[->,>=stealth',shorten >=1pt,auto,node distance=1.1cm,
        semithick,scale=1, every node/.style={scale=1}]
        \tikzstyle{every state}=[fill=white,draw=black,text=black,minimum size=3mm,inner sep=2pt]
        \node[state,align=center, fill=lightgray](A){A};
        \node[state,align=center,fill=lightgray](C)[right of=A]{C};
        \path (A) -- (C) coordinate[midway] (aux);
        \node[state,align=center, fill=lightgray, yshift=-0.5cm](AC)[below of=aux]{};
        \node[state,align=center, fill=YellowOrange](B)[right of=C]{B};
        \node[state,align=center, fill=YellowOrange](E)[right of=B]{E};
        \path (B) -- (E) coordinate[midway] (aux2);
        \node[state,align=center, fill=YellowOrange, yshift=-0.5cm](BE)[below of=aux2]{};
        \node[state,align=center](D)[right of=E]{D};
        \path (A) edge[draw=black] node[anchor=center,align=center, draw=lightgray, fill=lightgray,rounded rectangle] {\scriptsize2} (AC);
        \path (C) edge[draw=black] node[anchor=center,align=center, draw=lightgray, fill=lightgray,rounded rectangle] {\scriptsize2} (AC);
        \path (B) edge[draw=YellowOrange] node[anchor=center,align=center, draw=YellowOrange, fill=YellowOrange,rounded rectangle] {\scriptsize2} (BE);
        \path (E) edge[draw=YellowOrange] node[anchor=center,align=center, draw=YellowOrange, fill=YellowOrange,rounded rectangle] {\scriptsize2} (BE);
     \end{tikzpicture}
	\end{subfigure}
  \captionsetup{margin=2cm}
	\caption{Merge of \textcolor{ALUblue}{\{B\}} and \textcolor{ALUblue}{\{E\}}. On the left side, we see the distance matrix.
  On the right side we see the guide tree,
  where the orange colored nodes branches represent the changes made to the guide tree.}
\label{fig:ex1_5}
\end{figure}

\item \cref{fig:ex1_6} illustrates the distance matrices, before and after the merge of $\{B\}$ and $\{E\}$.
The distance between $\{B,E\}$ and the other clusters is the \textcolor{ALUred}{average distance between all their sequences}.
\begin{figure}[H]
  \centering
  \begin{subfigure}{.4\textwidth}
    \centering
    \begin{tabular}{c|c|c|c|c|}
      \cline{2-5}
      & \cellcolor[HTML]{FFFFFF}A,C & B & D & \cellcolor[HTML]{4B93C7}E \\ \hline
      \multicolumn{1}{|c|}{\cellcolor[HTML]{FFFFFF}A,C} & 0                           & 8 & 6 & 8                         \\ \hline
      \multicolumn{1}{|c|}{\cellcolor[HTML]{4B93C7}B}   & 8                           & 0 & 8 & \cellcolor[HTML]{F8A102}4\tikzmark{c} \\ \hline
      \multicolumn{1}{|c|}{D}                           & 6                           & 8 & 0 & 8                         \\ \hline
      \multicolumn{1}{|c|}{E}                           & 8                           & 4 & 8 & 0                         \\ \hline
    \end{tabular}
  \end{subfigure}
%%%%%%%%%%%%%%
  \begin{subfigure}{.4\textwidth}
    \centering
    \begin{tabular}{c|c|c|c|}
    \cline{2-4}
                                                      & \cellcolor[HTML]{FFFFFF}A,C & \cellcolor[HTML]{F8A102}B,E & D \\ \hline
    \multicolumn{1}{|c|}{\tikzmark{d}\cellcolor[HTML]{FFFFFF}A,C} & 0                           & 8                           & 6 \\ \hline
    \multicolumn{1}{|c|}{\cellcolor[HTML]{F8A102}B,E} & 8                           & 0                           & 8 \\ \hline
    \multicolumn{1}{|c|}{D}                           & 6                           & 8                           & 0 \\ \hline
    \end{tabular}
  \end{subfigure}
  \captionsetup{margin=2cm}
  \caption{Recalulation of distances, after the merge of \textcolor{ALUblue}{\{B\}} and \textcolor{ALUblue}{\{E\}}.
  On the left side, we see the old distance matrix.
  On the right side we see the new distance matrix, where \textcolor{ALUblue}{\{B\}} and \textcolor{ALUblue}{\{E\}} are merged.}
\begin{tikzpicture}[overlay,remember picture,->]
  \draw[very thick]         ($({pic cs:c})+(4.0ex,1ex)$)
      to   ($({pic cs:d})+(-4.8ex,-0.4ex)$);
  \end{tikzpicture}
  \label{fig:ex1_6}
\end{figure}

\item \cref{fig:ex1_7} illustrates this step of the algorithm.
\textcolor{ALUblue}{\{A,C\}} and \textcolor{ALUblue}{\{D\}} are the closest clusters,
the algorithm \textcolor{ALUgreen}{merges \{A,C\} and \{D\}  into \{A,C,D\}}.
Again a new node is introduced in the guide tree, to which we connect  $D$ and the common node of $A$ and $C$.
It assigns $\frac{d(\{A,C\}, \{D\})}{2} = \frac{6}{2} = 3$ total length to each branch.
Observe, that the 1 is obtained calculating $3-\textcolor{ALUred}{2} = 1$.
\begin{figure}[H]
	\centering
	\begin{subfigure}{.4\textwidth}
    \centering
    \begin{tabular}{c|c|c|c|}
    \cline{2-4}
                                                      & \cellcolor[HTML]{FFFFFF}A,C & \cellcolor[HTML]{FFFFFF}B,E & \cellcolor[HTML]{4B93C7}D \\ \hline
    \multicolumn{1}{|c|}{\cellcolor[HTML]{4B93C7}A,C} & 0                           & 8                           & \cellcolor[HTML]{F8A102}6 \\ \hline
    \multicolumn{1}{|c|}{\cellcolor[HTML]{FFFFFF}B,E} & 8                           & 0                           & 8                         \\ \hline
    \multicolumn{1}{|c|}{D}                           & 6                           & 8                           & 0                         \\ \hline
    \end{tabular}
	\end{subfigure}
%%%%%%%%%%%%%%
	\begin{subfigure}{.4\textwidth}
    \centering
    \begin{tikzpicture}[-,>=stealth',shorten >=1pt,auto,node distance=1.1cm,
        semithick,scale=1, every node/.style={scale=1}]
        \tikzstyle{every state}=[fill=white,draw=black,text=black,minimum size=3mm,inner sep=2pt]
        \node[state,align=center, fill=lightgray](A){A};
        \node[state,align=center,fill=lightgray](C)[right of=A]{C};
        \node[state,align=center, fill=YellowOrange](D)[right of=C]{D};
        \path (A) -- (C) coordinate[midway] (aux);
        \path (aux) -- (D) coordinate[midway] (aux3);
        \node[state,align=center, fill=YellowOrange, yshift=-0.5cm](AC)[below of=aux]{};
        \node[state,align=center, fill=YellowOrange, yshift=-1.5cm](ACD)[below of=aux3]{};
        \node[state,align=center, fill=lightgray](B)[right of=D]{B};
        \node[state,align=center, fill=lightgray](E)[right of=B]{E};
        \path (B) -- (E) coordinate[midway] (aux2);
        \node[state,align=center, fill=lightgray, yshift=-0.5cm](BE)[below of=aux2]{};
        \path (A) edge[draw=black] node[anchor=center,align=center, draw=lightgray, fill=lightgray,rounded rectangle] {\scriptsize\textcolor{ALUred}{2}} (AC);
        \path (C) edge[draw=black] node[anchor=center,align=center, draw=lightgray, fill=lightgray,rounded rectangle] {\scriptsize\textcolor{ALUred}{2}} (AC);

        \path (AC) edge[draw=YellowOrange] node[anchor=center,align=center, draw=YellowOrange, fill=YellowOrange,rounded rectangle] {\scriptsize1} (ACD);
        \path (D) edge[draw=YellowOrange] node[anchor=center,align=center, draw=YellowOrange, fill=YellowOrange,rounded rectangle] {\scriptsize3} (ACD);

        \path (B) edge[draw=black] node[anchor=center,align=center, draw=lightgray, fill=lightgray,rounded rectangle] {\scriptsize2} (BE);
        \path (E) edge[draw=black] node[anchor=center,align=center, draw=lightgray, fill=lightgray,rounded rectangle] {\scriptsize2} (BE);
     \end{tikzpicture}
	\end{subfigure}
  \captionsetup{margin=2cm}
	\caption{Merge of \textcolor{ALUblue}{\{A,C\}} and \textcolor{ALUblue}{\{D\}}. On the left side, we see the distance matrix.
  On the right side we see the guide tree,
  where the orange colored nodes branches represent the changes made to the guide tree.}
\label{fig:ex1_7}
\end{figure}
\item \cref{fig:ex1_8} illustrates the distance matrices, before and after the merge of $\{A,C\}$ and $\{D\}$.
The distance between \textcolor{ALUblue}{$\{A,C,D\}$} and \textcolor{ALUblue}{$\{B,E\}$} is the \textcolor{ALUred}{average distance between all their sequences}.
\tikzset{box/.style={inner xsep=0pt}}
\begin{figure}[H]
  \centering
  \begin{subfigure}{.4\textwidth}
    \centering
    \begin{tabular}{c|c|c|c|}
      \cline{2-4}
      & \cellcolor[HTML]{FFFFFF}A,C & \cellcolor[HTML]{FFFFFF}B,E & \cellcolor[HTML]{4B93C7}D \\ \hline
      \multicolumn{1}{|c|}{\cellcolor[HTML]{4B93C7}A,C} & 0                           & 8                           & \cellcolor[HTML]{F8A102}6 \\ \hline
      \multicolumn{1}{|c|}{\cellcolor[HTML]{FFFFFF}B,E} & 8                           & 0                           & 8\tikzmark{e}                         \\ \hline
      \multicolumn{1}{|c|}{D}                           & 6                           & 8                           & 0                         \\ \hline
    \end{tabular}
  \end{subfigure}
%%%%%%%%%%%%%%
  \begin{subfigure}{.4\textwidth}
    \centering
    \begin{tabular}{c|c|c|}
    \cline{2-3}
                                                        & \cellcolor[HTML]{F8A102}A,C,D & B,E \\ \hline
    \multicolumn{1}{|c|}{\tikzmark{f}\cellcolor[HTML]{F8A102}A,C,D} & 0                             & 8   \\ \hline
    \multicolumn{1}{|c|}{\cellcolor[HTML]{FFFFFF}B,E}   & 8                             & 0   \\ \hline
    \end{tabular}
  \end{subfigure}
  \captionsetup{margin=2cm}
  \caption{Recalulation of distances, after the merge of \textcolor{ALUblue}{\{A,C\}} and \textcolor{ALUblue}{\{D\}}.
  On the left side, we see the old distance matrix.
  On the right side we see the new distance matrix, where \textcolor{ALUblue}{\{A,C\}} and \textcolor{ALUblue}{\{D\}} are merged.}
\label{fig:ex1_8}
\end{figure}
\begin{tikzpicture}[overlay,remember picture,->]
  \draw[very thick]         ($({pic cs:e})+(4.0ex,1ex)$)
      to   ($({pic cs:f})+(-4.8ex,-0.4ex)$);
  \end{tikzpicture}

  \item \cref{fig:ex1_9} illustrates this step of the algorithm.
  \textcolor{ALUblue}{\{A,C, D\}} and \textcolor{ALUblue}{\{B,E\}} are remaining.
  The algorithms \textcolor{ALUgreen}{merges \{A,C,D\} and \{B,E\}  into \{A,C,D, B, E\}}.
  In the guide tree a new node is introduced, the node which connects $A,C$ and $D$ and the node which connects $B$ and $E$,
  are connected to it. The algorithm assigns $\frac{d(\{A,C, D\}, \{B,E\})}{2} = \frac{8}{2} = 4$ total length to each branch.
  Observe, that the 1 is obtained calculating $4-\textcolor{ALUred}{3} = 1$.
  Observe, that the 2 is obtained calculating $4-\textcolor{ALUgreen}{2} = 2$.
  \begin{figure}[H]
  	\centering
  	\begin{subfigure}{.4\textwidth}
      \centering
      \begin{tabular}{c|c|c|}
      \cline{2-3}
                                                          & A,C,D & \cellcolor[HTML]{4B93C7}B,E \\ \hline
      \multicolumn{1}{|c|}{\cellcolor[HTML]{4B93C7}A,C,D} & 0     & \cellcolor[HTML]{F8A102}8   \\ \hline
      \multicolumn{1}{|c|}{\cellcolor[HTML]{FFFFFF}B,E}   & 8     & 0                           \\ \hline
      \end{tabular}
  	\end{subfigure}
  %%%%%%%%%%%%%%
  	\begin{subfigure}{.4\textwidth}
      \centering
      \begin{tikzpicture}[-,>=stealth',shorten >=1pt,auto,node distance=1.1cm,
          semithick,scale=1, every node/.style={scale=1}]
          \tikzstyle{every state}=[fill=white,draw=black,text=black,minimum size=3mm,inner sep=2pt]
          \node[state,align=center, fill=lightgray](A){A};
          \node[state,align=center,fill=lightgray](C)[right of=A]{C};
          \node[state,align=center, fill=lightgray](D)[right of=C]{D};
          \path (A) -- (C) coordinate[midway] (aux);
          \path (aux) -- (D) coordinate[midway] (aux3);
          \node[state,align=center, fill=lightgray, yshift=-0.5cm](AC)[below of=aux]{};
          \node[state,align=center, fill=YellowOrange, yshift=-1.5cm](ACD)[below of=aux3]{};
          \node[state,align=center, fill=lightgray](B)[right of=D]{B};
          \node[state,align=center, fill=lightgray](E)[right of=B]{E};
          \path (B) -- (E) coordinate[midway] (aux2);
          \node[state,align=center, fill=YellowOrange, yshift=-0.5cm](BE)[below of=aux2]{};
          \node[state,align=center, fill=YellowOrange, yshift=-2.5cm](ACDBE)[below of=D]{};

          \path (A) edge[draw=black] node[anchor=center,align=center, draw=lightgray, fill=lightgray,rounded rectangle] {\scriptsize2} (AC);
          \path (C) edge[draw=black] node[anchor=center,align=center, draw=lightgray, fill=lightgray,rounded rectangle] {\scriptsize2} (AC);

          \path (AC) edge[draw=black] node[anchor=center,align=center, draw=lightgray, fill=lightgray,rounded rectangle] {\scriptsize1} (ACD);
          \path (D) edge[draw=black] node[anchor=center,align=center, draw=lightgray, fill=lightgray,rounded rectangle] {\scriptsize\textcolor{ALUred}{3}} (ACD);

          \path (B) edge[draw=black] node[anchor=center,align=center, draw=lightgray, fill=lightgray,rounded rectangle] {\scriptsize\textcolor{ALUgreen}{2}} (BE);
          \path (E) edge[draw=black] node[anchor=center,align=center, draw=lightgray, fill=lightgray,rounded rectangle] {\scriptsize\textcolor{ALUgreen}{2}} (BE);
          \path (ACD) edge[draw=YellowOrange] node[anchor=center,align=center, draw=YellowOrange, fill=YellowOrange,rounded rectangle] {\scriptsize1} (ACDBE);
          \path (BE) edge[draw=YellowOrange] node[anchor=center,align=center, draw=YellowOrange, fill=YellowOrange,rounded rectangle] {\scriptsize2} (ACDBE);
       \end{tikzpicture}
  	\end{subfigure}
    \captionsetup{margin=2cm}
  	\caption{Merge of \textcolor{ALUblue}{\{A,C,D\}} and \textcolor{ALUblue}{\{B,E\}}. On the left side, we see the distance matrix.
    On the right side we see the guide tree, where the orange colored nodes branches represent the changes made to the guide tree.}
\label{fig:ex1_9}
  \end{figure}
\item There are no more clusters to merge, the algorithm terminates.
\cref{fig:ex1_10} illustrates the final state of the distance matrix and guide tree.
\begin{figure}[H]
  \centering
  \begin{subfigure}{.4\textwidth}
    \centering
    \begin{tabular}{c|c|}
    \cline{2-2}
                                                            & \cellcolor[HTML]{F8A102}A,C,D,B,E \\ \hline
    \multicolumn{1}{|c|}{\cellcolor[HTML]{F8A102}A,C,D,B,E} & 0                                 \\ \hline
    \end{tabular}
  \end{subfigure}
%%%%%%%%%%%%%%
  \begin{subfigure}{.4\textwidth}
    \centering
    \begin{tikzpicture}[-,>=stealth',shorten >=1pt,auto,node distance=1.1cm,
      semithick,scale=1, every node/.style={scale=1.0}]
      \tikzstyle{every state}=[fill=white,draw=black,text=black,minimum size=3mm,inner sep=2pt]
      \node[state,align=center, fill=lightgray](A){A};
      \node[state,align=center,fill=lightgray](C)[right of=A]{C};
      \node[state,align=center, fill=lightgray](D)[right of=C]{D};
      \path (A) -- (C) coordinate[midway] (aux);
      \path (aux) -- (D) coordinate[midway] (aux3);
      \node[state,align=center, fill=lightgray, yshift=-0.5cm](AC)[below of=aux]{};
      \node[state,align=center, fill=lightgray, yshift=-1.5cm](ACD)[below of=aux3]{};
      \node[state,align=center, fill=lightgray](B)[right of=D]{B};
      \node[state,align=center, fill=lightgray](E)[right of=B]{E};
      \path (B) -- (E) coordinate[midway] (aux2);
      \node[state,align=center, fill=lightgray, yshift=-0.5cm](BE)[below of=aux2]{};
      \node[state,align=center, fill=lightgray, yshift=-2.5cm](ACDBE)[below of=D]{};

      \path (A) edge[draw=black] node[anchor=center,align=center, draw=lightgray, fill=lightgray,rounded rectangle] {\scriptsize2} (AC);
      \path (C) edge[draw=black] node[anchor=center,align=center, draw=lightgray, fill=lightgray,rounded rectangle] {\scriptsize2} (AC);

      \path (AC) edge[draw=black] node[anchor=center,align=center, draw=lightgray, fill=lightgray,rounded rectangle] {\scriptsize1} (ACD);
      \path (D) edge[draw=black] node[anchor=center,align=center, draw=lightgray, fill=lightgray,rounded rectangle] {\scriptsize3} (ACD);

      \path (B) edge[draw=black] node[anchor=center,align=center, draw=lightgray, fill=lightgray,rounded rectangle] {\scriptsize2} (BE);
      \path (E) edge[draw=black] node[anchor=center,align=center, draw=lightgray, fill=lightgray,rounded rectangle] {\scriptsize2} (BE);
      \path (ACD) edge[draw=black] node[anchor=center,align=center, draw=lightgray, fill=lightgray,rounded rectangle] {\scriptsize1} (ACDBE);
      \path (BE) edge[draw=black] node[anchor=center,align=center, draw=lightgray, fill=lightgray,rounded rectangle] {\scriptsize2} (ACDBE);
    \end{tikzpicture}
  \end{subfigure}
  \captionsetup{margin=2cm}
  \caption{On the left side, we see the final distance matrix. On the right side we see the final guide tree}
\label{fig:ex1_10}
\end{figure}
\end{enumerate}
\end{example}
