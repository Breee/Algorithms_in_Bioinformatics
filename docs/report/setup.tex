
\usepackage{lmodern}
\renewcommand{\sfdefault}{lmss}
\renewcommand{\ttdefault}{lmtt}

\usepackage[T1]{fontenc}
\usepackage[latin9]{inputenc}
\usepackage{ textcomp }
\usepackage{ stmaryrd }
\usepackage{fancyhdr}
\pagestyle{fancy}
\setcounter{secnumdepth}{3}
\setlength{\parskip}{\medskipamount}
\setlength{\parindent}{0pt}
\usepackage{babel}
\usepackage{amsmath}
\usepackage{amssymb}
\usepackage{graphicx}
\usepackage[unicode=true,
bookmarks=true,bookmarksnumbered=true,bookmarksopen=true,bookmarksopenlevel=1,
breaklinks=false,pdfborder={0 0 0},backref=false,colorlinks=true]{hyperref}

\hypersetup{pdftitle={Generation of commutativity conditions},
	pdfauthor={Julian Loeffler},
	pdfsubject={Generation of commutativity conditions},
	pdfkeywords={},
	pdfpagelayout=OneColumn, pdfnewwindow=true, pdfstartview=XYZ, plainpages=false,
	colorlinks=true,linkcolor=red}
	% ------------------- functional, default-------------------
	\usepackage[dvipsnames]{xcolor}
	\usepackage{color, colortbl}
	\usepackage{fancyvrb}
	\usepackage{array}  % custom format per column in table - needed on the title page
	\usepackage{graphicx}  % include graphics
	\usepackage{subcaption}
	\usepackage{amsmath}  % |
	\usepackage{amsthm}   % | math, bmatrix etc
	\usepackage{amsfonts} % |
	\usepackage{amssymb}
	\usepackage{calc}  % calculate within LaTeX
	\usepackage{enumitem}
	\usepackage{fancyvrb}
	\usepackage{pdfpages}
	\usepackage{ltablex}
	\usepackage{tabularx}
	%tablularx columntype
	\newcolumntype{s}{>{\hsize=.25\hsize}X}

	\usepackage{numprint}
	\usepackage{float}
	\usepackage{dirtree}
	\makeatletter

%%%%%%%%%%%%%%%%%%%%%%%%%%%%%% LyX specific LaTeX commands.
\pdfpageheight\paperheight
\pdfpagewidth\paperwidth


\@ifundefined{date}{}{\date{}}

%%%%%%%%%%%%%%%%%%%%%%%%%%%%%% User specified LaTeX commands.
% Linkflaeche für Querverweise vergrößern und automatisch benennen
\AtBeginDocument{\renewcommand{\ref}[1]{\mbox{\autoref{#1}}}}
\newlength{\abc}
\settowidth{\abc}{\space}
\AtBeginDocument{%
	\addto\extrasngerman{
		\renewcommand{\equationautorefname}{\hspace{-\abc}}
		\renewcommand{\sectionautorefname}{Kap.\negthinspace}
		\renewcommand{\subsectionautorefname}{Kap.\negthinspace}
		\renewcommand{\subsubsectionautorefname}{Kap.\negthinspace}
		\renewcommand{\figureautorefname}{Abb.\negthinspace}
		\renewcommand{\tableautorefname}{Tab.\negthinspace}
	}
}

% für den Fall, dass jemand die Bezeichnung "Gleichung" haben will
%\renewcommand{\eqref}[1]{equation~(\negthinspace\autoref{#1})}

% Setzt den Link für Sprünge zu Gleitabbildungen
% auf den Anfang des Gleitobjekts und nicht aufs Ende
\usepackage[figure]{hypcap}

% Die Seiten des Inhaltsverzeichnisses werden römisch numeriert,
% ein PDF-Lesezeichen für das Inhaltsverzeichnis wird hinzugefügt
\let\myTOC\tableofcontents
\renewcommand\tableofcontents{%
	\pdfbookmark[1]{\contentsname}{}
	\myTOC
}
% Fixes table of content spacings. e.g. 10.10blubb becomes 10.10 blubb
\usepackage[tocindentauto]{tocstyle}
\usetocstyle{standard}

%footnote URL
\usepackage{url}

%code
\usepackage{listings}
\usepackage{accsupp}

\lstdefinestyle{customc}{
  belowcaptionskip=1\baselineskip,
  breaklines=true,
	numbers=left,
  stepnumber=1,
  frame=single,
	columns=flexible,
  keepspaces=true,
  xleftmargin=\parindent,
  language=C,
  showstringspaces=false,
  basicstyle=\footnotesize\ttfamily,
  keywordstyle=\bfseries\color{green!40!black},
  commentstyle=\itshape\color{purple!40!black},
  identifierstyle=\color{blue},
  stringstyle=\color{orange},
	captionpos=b,
	belowcaptionskip=0.1cm,
	belowskip=0.1cm,
	numberstyle=\noncopynumber
}

\definecolor{Black}{gray}{0.0}
\definecolor{White}{gray}{0.9}

\lstdefinestyle{bash}{
  belowcaptionskip=1\baselineskip,
  breaklines=true,
	numbers=left,
  stepnumber=1,
  frame=single,
	columns=flexible,
  keepspaces=true,
  xleftmargin=\parindent,
  showstringspaces=false,
  basicstyle=\footnotesize\ttfamily\color{black},
  keywordstyle=\bfseries\color{black},
  commentstyle=\itshape\color{black},
  identifierstyle=\color{black},
  stringstyle=\color{black},
	captionpos=b,
	belowcaptionskip=0.1cm,
	belowskip=0.1cm,
	numberstyle=\noncopynumber,
	deletekeywords = {enable,command,signed,unsigned,float}
}

\newcommand{\noncopynumber}[1]{%
    \BeginAccSupp{method=escape,ActualText={}}%
    #1%
    \EndAccSupp{}%
}

\lstset{style=customc}
%\renewcommand{\lstlistingname}{Code}

\usepackage{xspace}
%------------------------------------------------------------------------------
%       (re)new commands / settings
%------------------------------------------------------------------------------
% ----------------- referencing ----------------

\newcommand{\classname}[1]{\texttt{#1}}
\newcommand{\verifyerror}{\texttt{\_\_VERIFIER\_error()}\xspace}
\newcommand{\nondetfloat}{\texttt{\_\_VERIFIER\_nondet\_float()}\xspace}
\newcommand{\nondetdouble}{\texttt{\_\_VERIFIER\_nondet\_double()}\xspace}
\newcommand{\precond}{\texttt{\_\_VERIFIER\_precond\_reach()}\xspace}
\newcommand{\ultimate}{\textsc{Ultimate}\xspace}
\newcommand{\automizer}{\textsc{Ultimate Automizer}\xspace}
\newcommand{\framac}{\textsc{Frama-C}\xspace}
\newcommand{\cbmc}{\textsc{CBMC}\xspace}
\newcommand{\clang}{\textsc{Clang Static Analyzer}\xspace}
\newcommand{\infer}{\textsc{Facebook Infer}\xspace}
\newcommand{\coverity}{\textsc{Coverity}\xspace}
\newcommand{\nox}{\textsc{Nox-Executor}\xspace}
\newcommand{\safe}{\texttt{SAFE}\xspace}
\newcommand{\unsafe}{\texttt{UNSAFE}\xspace}
\newcommand{\timeout}{\texttt{TIMEOUT}\xspace}
\newcommand{\error}{\texttt{ERROR}\xspace}
\newcommand{\unknown}{\texttt{UNKNOWN}\xspace}
\newcommand{\trans}[1]{\llparenthesis #1 \rrparenthesis}
\newcommand{\liftedtrans}[1]{\mathit{trans}(#1)}
\newcommand{\multilineequation}[1]{\Bigl(\begin{matrix}#1\end{matrix}\Bigr)}
\newcommand{\servois}{\textsc{Servois}\xspace}
\newcommand{\refine}{\texttt{refine}\xspace}
\newcommand{\refinefunc}[1]{\texttt{refine}(#1)\xspace}
\newcommand{\lift}{\texttt{lift}\xspace}
\newcommand{\choosecmd}{\texttt{choose}\xspace}
\newcommand{\choosecmdfunc}[1]{\texttt{choose}(#1)\xspace}
\newcommand{\true}{\mathit{true}\xspace}
\newcommand{\false}{\mathit{false}\xspace}

% ------------------- marker commands -------------------
% ToDo command
%\newcommand{\todo}[1]{\textbf{\textcolor{red}{(TODO: #1)}}}
\newcommand{\extend}[1]{\textbf{\textcolor{green}{(EXTEND: #1)}}}
% Lighter color to note down quick drafts
%\newcommand{\draft}[1]{\textbf{\textcolor{NavyBlue}{(DRAFT: #1)}}}

\usepackage[%
%    disable%
]{todonotes}

% wenn man disable einkommentiert, verschwinden alle TODOs aus dem Text.
%% some todo commands
\newcommand{\dd}[1]{\todo[color=green!40]{#1}\xspace}
\newcommand{\mg}[1]{\todo[color=blue!40]{#1}\xspace}
\newcommand{\jl}[1]{\todo[color=black!40, size=\tiny]{JL: #1}\xspace}
\newcommand{\jlin}[2]{\todo[inline,color=black!40,caption={#1}]{JL: #2}}
\newcommand{\jlins}[1]{\jlin{#1}{#1}}
\newcommand{\todoin}[1]{\todo[inline]{#1}}
\newcommand{\secnotes}[1]{\todo[inline,color=cyan!40]{#1}\xspace}
\newcommand{\cn}{\todo{Citation needed}\xspace}
\newcommand{\ddin}[2]{\todo[inline,color=green!40,caption={#1}]{#2}}
\newcommand{\ddins}[1]{\ddin{#1}{#1}}
\newcommand{\mgin}[2]{\todo[inline,color=blue!40,caption={#1}]{#2}}
\newcommand{\mgins}[1]{\mgin{#1}{#1}}


\usepackage[noabbrev,capitalize]{cleveref}

\newtheoremstyle{break}
  {\topsep}{\topsep}%
  {\itshape}{}%
  {\bfseries}{}%
  {\newline}{}%
\theoremstyle{break}
\newtheorem{definition}{Definition}
\newtheorem{example}{Example}
\usepackage{mathtools}
\usepackage[]{algorithmicx}
\usepackage[]{algorithm}
\usepackage{algpseudocode}


\usepackage{tikz}
\usetikzlibrary{shapes.misc, positioning}
\usetikzlibrary{calc, quotes, tikzmark, shapes,arrows,automata}

\newcommand{\figref}{Figure \ref}

\definecolor{gray}{rgb}{0.4,0.4,0.4}
\definecolor{darkblue}{rgb}{0.0,0.0,0.6}
\definecolor{cyan}{rgb}{0.0,0.6,0.3}
\definecolor{ALUblue}{rgb}{0,.42,.714}
\colorlet{ALUred}{red!70!black}
\definecolor{ALUgreen}{rgb}{.1,.5,0}
\definecolor{ALUblue}{rgb}{0,.42,.714}
\colorlet{ALUgray}{white!95!ALUblue}
